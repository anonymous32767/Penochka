\documentclass[a4papre,12pt]{article}
\usepackage[utf8]{inputenc}
\usepackage[english,russian]{babel}
\title{Некоторые специфические замеры скорости яваскрипта}
\date{\today}
\begin{document}

\maketitle

\begin{abstract}
Описывается небольшое исследование производительности браузерного
яваскрипта, которое потребовалось чтобы понять узкие места, удостовериться
в паре предположений и сделать выбор между selectors api и xpath.

Целью исследования является выяснение средней скорости работы в достаточно
современном браузере отдельных конструкций яваскрипта. Маловероятно, что
результаты настоящего тестирования позволяют судить объективно о
превосходстве скоростных возможностей одного браузера над другим.
\end{abstract}

\section{Цели работы}

При проектировании новой версии пеночки возникли типичные проблемы
проектирования: стало непонятно, как поведут себя некоторые паттерны,
которые планировалось часто использовать. В частности для сокрытия
информации в модулях использовалась следующая конструкция:

\begin{verbatim}
;(function () {
   ...module code here...
})();
\end{verbatim}

Она конечно слишком простая для того, чтобы тормозить, однако практика
показала, что с развитием скрипта количество возможностей в нем, а значит и
количество модулей, имеет тенденцию переходить все разумные
пределы. Поэтому стало страшно и захотелось померять скорость. 

Также в ходе безумных исследований интернетов с помощью гугла было
установлено, что в браузерах на сегодняшний день существуют две примерно
одинаковые технологии выборки нужного из DOM'a --- CSS-селекторы
(функции querySelector, querySelectorAll) и XPath (в основном функция
document.evaluate и некоторые другие). Автоматически встала проблема выбора
какой-либо одной технологии, в ходе решения которой данные о скорости
выборки обоими методами были бы не лишни.

В-третьих обнаружилось, что родные браузерные события --- довольно
медленная штука, не предназначенная для того, чтобы ими кидались как
попало, а кидаться как попало было надо: OOD помимо объектов, имеющих
внутреннее, а значит скрытое состояние подразумевает еще и возможность
обмена сообщениями. Поэтому нужен механизм обмена сообщениями, причем
быстрый механизм. Чтобы окончательно удостовериться в тормознутости
нативных событий браузеров нужно тоже замерить скорость их выполнения.

Наконец, несколько замеров производительности дадут некую общую температуру
по палате: можно будет примерно понять, чего следует ожидать от яваскрипта.

\section{Методика исследования}

Для тестирования производительности использовалась экспериментальный
прототип новой версии пеночки. Этим убивалось три зайца: (1) тестировалась
производительность, (2) тестировалась производительностьи именно пеночки и
(3) обкатывались её несущие конструкции. Исходник для тестирования приведен
в приложении.

Все тестирование состоит из пяти тестов:
\begin{description}
\item[Namespace] Тест конструкции предназначенной для сокрытия в модуле. В
  цикле такая конструкция создавалась миллион раз, время чего и замерялось.
\item[Penochka msgs] Собственная реализация событий (подробности см. в
  исходнике, модуль events). В хэше создавалось порядка 100 событий (чтобы
  браузеру было из чего выбирать), на какое-то одно вешалось 10
  обработчиков и потом оно вызывалось 100000 раз. То есть опять таки
  миллион итераций.
\item[Native msgs] Браузерная реализация событий. Алгоритм такой же, как и
  в случае тестировани собственной реализации, только дополнительно 100 
  событий не создавалось --- в браузере и своих хватает.
\item[Xpath selection] Выборка из ДОМа с помощью XPath. Из заглавной
  страницы сайта lenta.ru 1000 раз выбирались все гиперссылки (селектор
  {\tt '//a'}).

\item[CSS selection] Выборка из ДОМа с помощью CSS. Из заглавной страницы
  lenta.ru также 1000 раз выбирались все гиперссылки (селектор {\tt 'a'}).
\end{description}

Производительность исследовалась на 7 браузерах: Опере 10.5, Опере 10.10,
Хроме, Хромиуме, ФФ 3.0, ФФ 3.5, ФФ 3.6.

Измерения проводились на компьютере следующей конфигурации:

\begin{verbatim}
Название ОС:                      Microsoft Windows 7 Максимальная
Версия ОС:                        6.1.7600 Н/Д сборка 7600
Сборка ОС:                        Multiprocessor Free
Тип системы:                      x64-based PC
Процессор(ы):                     Число процессоров - 1.
                                  [01]: AMD64 Family 
Полный объем физической памяти:   1.791 МБ
\end{verbatim}

В ходе измерений на том же компьютере играла музычка (ogg vorbis и mp3).

\section{Результаты}

Результаты сведены в следующую табличку. Преведено среднее время выполнения
заданного теста заданным браузеров в миллисекундах.\\
\\
\begin{tabular}{l|r r r r r r r r }
\textbf{Browser} & \textbf{O 10.5} & \textbf{O 10.10} & \textbf{Chr} &
\textbf{Chrm} & \textbf{FF3.0} & \textbf{FF3.5} & \textbf{FF3.6} \\
\hline
Namespace       & 725   & 1747  & 250   & 310   & 4000  & 2200  & 1950 \\
Penochka msgs   & 808   & 2080  & 530   & 940   & 1900  & 1500  & 1300 \\
Native msgs     & E1    & E1    & 3950  & 4300  & 9000  & 6700  & 5600 \\
Xpath selection & 1160  & 3400  & 1200  & 1280  & 7500  & 4100  & 3700 \\
CSS selection   & 1380  & 3200  & 1150  & 1210  & NS    & 4800  & 5100 \\
\end{tabular}\\
\\
Если в табличке число оканчивается на круглую цифру (нуль), то это
означает, что в этих разрядах цифры варьировались от измерения к измерению,
то есть по сути это отклонения из-за разного рода погрешностей.

Названия браузеров означают: \textbf{O 10.5} --- Opera 10.5 (сборка 3172),
\textbf{O 10.10} --- Opera 10.10 (сборка 1893), \textbf{Chr} --- Google
Chrome (версия 3.0.195.38), \textbf{Chrm} --- Chromium 4.0.302.0 (сборка
36495), \textbf{FF3.0} --- Mozilla Firefox 3.0.13, \textbf{FF3.5} ---
Mozilla Firefox 3.5.7, \textbf{FF3.6} --- Namoroka 3.6a1.

Если в ячейке таблицы стоит не число, а аббревиатура, то произошло
следующее:
\begin{description}
\item[NS] Нужный функционал не поддерживается.
\item[E1] В ходе тестирования Опера планомерно съела всю физическую память
  и вылетела с ошибкой <<Недостаточно памяти>>, даже не предложив отправить
  отчет об ошибке.
\end{description}

\section{Анализ}

Проведенное тестирование позволяет говорить о:
\begin{enumerate}
\item Скорость создания конструкции, скрывающей информацию достаточна для
  использования и не должна стать причиной падения производительности.
\item Собственная реализация событий всегда быстрее браузерного варианта в
  ~4 раза. Также падение оперы в ходе тестирования говорит о том, что
  использовать эти события кроме как для подписки на всякие {\tt 'click'} или
  {\tt 'mousemove'} не стоит. Хотя с другой стороны такие плачевные
  результаты могут намекать на то, что методика измерений была неверная и
  кое-кто не умеет обращаться с браузерными событиями как следует.
\item Селекторы XPath и CSS имеют примерно одинаковую и достаточную для
  использования производительность. Вместе с тем CSS селекторы не
  поддерживаются Firefox'ом версии 3.0, что не добавляет им очков.
\item Хоть точность результатов и достаточна, скрипт тестирования содержит
  потенциал для получения более точных значений. Его улучшение является
  вопросом будующего.
\end{enumerate}

\newpage

\section*{Приложение. Исходные текст скрипта для тестирования}

\begin{verbatim}
/** kernel - Основные функции

    Определение глобального объекта пеночки. Некоторые, удобные в
использовании функции, аналогичные по своей сущности библиотеке
jQuery. */
;(function () {
var window = this,
       undefined,
       pn = window.pn = function (el) {
       	  return new pn.fn.init(el)
       }

       pn.fn = pn.prototype = {
          init: function (el) {
	     if (el && el.fn)
	     	return el

	     return pn.fn.merge([el ? el : document])
	  },
	  merge: function (arr) {
	     this.length = 0;
	     if (toString.call(arr) === "[object Array]") {
	     	Array.prototype.push.apply(this, arr)
	     } else {
	     	Array.prototype.push.apply(this, [arr])
	     }
	     return this
	  }
       }
       
       pn.extend = pn.fn.extend = function (code) {
       	  for (method_name in code) {
	     pn.fn[method_name] = code[method_name]
	  }
       }
})();

/** events - Синхронные события

    Механизм подписки на системные события браузера и обмена
внутренними сообщениями */
;(function () {
	var events = {}

	 pn.on = pn.fn.on = function(evname, fun, issys, iscap) {
	    if (issys != null) {
	       var dispatcher = this[0] || document
	       dispatcher.addEventListener(evname, fun, iscap)
	    } else {
	       try {
	       	  events[evname].push(fun)
	       } catch (err) {
		  events[evname] = [fun]
	       }
	    }
	 },
	 pn.to = pn.fn.to = function (evname, cookie) {
	    try {
	       for(var i = 0; i < events[evname].length; i++)
	          events[evname][i](cookie)
	    } catch (err) {}
	 }
   
})();

/** xquery - XPath query

    Выборки элементов из DOM посредством xpath */
;(function () {

      pn.extend({
	init: function (selector) {
	   if (selector && selector.fn) 
	      return selector

	    if (typeof selector === "string") 
	       return pn(document).xpfind(selector)

	   return pn.fn.merge(selector || document)
	},
	xpfind: function (what) {
	   var xr = document.evaluate(what, this[0], 
         null, XPathResult.ANY_TYPE, null)
	   var e = xr.iterateNext()
	   this.length = 0

	   while (e) {
	      Array.prototype.push.apply(this, [e])
	      e = xr.iterateNext()
	   }

	   return this
	},	
      })

   pn.fn.find = pn.fn.xpfind
		 
   
})();

/** cquery - CSS query

    Выборки элементов из DOM посредством селекторов CSS */
;(function () {

	     var Sizzle = function (selector, context) {
			  return context.querySelectorAll(selector)
		  }
	  

      pn.extend({
	init: function (selector) {
	   if (selector && selector.fn) 
	      return selector

	    if (typeof selector === "string") 
	       return pn(document).cssfind(selector)

	   return pn.fn.merge(selector || document)
	},
	cssfind: function (what) {
	   this.length = 0
		var cr = Sizzle(what, this[0])
		
		for (var i = 0; i < cr.length; i++) {
	   	 Array.prototype.push.apply(this, [cr[i]])
		}

	   return this
	}	
      })

   pn.fn.find = pn.fn.cssfind 
   
})();

/** speedtest - Тесты скорости яваскрипта

    Тесты производительности специфических для пеночки конструкций */
;(function () {

       function time (fun) {
          var start = new Date().getTime()
   	  fun()
   	  return (new Date().getTime()) - start
       }

       var TEST_ITERATIONS = 1000000;

       window.namespace_create_test = function () {
       	  var a = 0
   	  var test = function () {
      	  for (var i = 0; i < TEST_ITERATIONS; i++) {
	        (function () {
	     	   a = a + 1
	        })()
      	     } 
       	  }

       	  var retval = time(test);
       	  return a == TEST_ITERATIONS ? retval : 'test failed';
       }

       window.message_pass_test = function () {
          var events = {}

   	  var AVG_EVENTS = 100;
   	  var AVG_HANDLERS = 10;

   	  var a = 0;

   	  for (var i = 0; i < AVG_EVENTS; i++) {
      	     pn.on('event'+ i, function () {})
   	  }
   
	  for (var i = 0; i < AVG_HANDLERS; i++) {
      	     pn.on('event10', function(e) { a = a + 1 })
   	  }

   	  function test () {
      	     for (var i = 0; i < TEST_ITERATIONS / AVG_HANDLERS; i++) {
	     	pn.to('event10', null)
      	     }
   	  }
   
	   var retval = time(test);
   	   return a == TEST_ITERATIONS ? retval : 'test failed';
	}

	window.nativemessage_pass_test = function () { 
  	   var AVG_HANDLERS = 10;

   	   var a = 0;

   	   for (var i = 0; i < AVG_HANDLERS; i++) {
      	      document.addEventListener('event10', 
                  function(e) { a = a + 1 }, true)
   	   }

   	   function test () {
      	      for (var i = 0; i < TEST_ITERATIONS / AVG_HANDLERS; i++) {
	         var evObj = document.createEvent('Events');
	 	 evObj.initEvent('event10', false, false)
	 	 document.dispatchEvent(evObj)
      	      }
   	   }
   
	   var retval = time(test);
   	   return a == TEST_ITERATIONS ? retval : 'test failed';
   }

	window.xpath_test = function () {
	   var DIVIDER = 1000
		var a = null

		function test () {
         for (var i = 0; i < TEST_ITERATIONS / DIVIDER; i++) {
	   		 a = pn(document).xpfind('//a')
         }
   	}

		var retval = time(test);
   	return a ? retval : 'test failed';
	}

	window.selectors_test = function () {
	   var DIVIDER = 1000
		var a = null

		function test () {
         for (var i = 0; i < TEST_ITERATIONS / DIVIDER; i++) {
	   		 a = pn(document).cssfind('a')
         }
   	}

		var retval = time(test);
   	return a ? retval : 'test failed';
	}

   
})();
\end{verbatim}

\end{document}
