\documentclass[12pt,a4paper]{article}
\usepackage[T2A]{fontenc}
\usepackage[utf8]{inputenc}
\usepackage[russian]{babel}
\title{Кулуары пеночки}
\date{\today}
\begin{document}
\maketitle
\section*{Причины появления пеночки}

Пеночка зародилась на новом дваче. Сначала все сидели на теплом ламповом дваче, а после того как он закончился перебрались на нольчан, добрачан, иичан и подобные имиджборды. На нольчане очень нравился новый способ троллинга, который условно можно назвать "Запостил сажу и скрыл тред". С одной стороны эта фраза вызывает у ОПа и его сторонников неугодного треда известную степень неудовольствия, с другой она никаким образом не участвует в дискуссии то есть "не кормит" (кроме вырожденных случаев, встречающихся иногда в /a/).

Да и скрывать треды по мнению некоторых --- это приятно и полезно. Так как даже если их и стараться игнорировать, то глаз во время пролистывания все равно может зацепиться и привести к нежелательному восприятию.

Поэтому, после некоторого, достаточно долгого и интенсивного сидения на новом дваче, невольно начали вспоминаться разного рода юзер-скрипты. Первым, как это ни странно, вспомнились эврикоскрипты, хотя их автор в последнее время и старался спрататься от толпы в некиих уютных местах. По мнению некоторых это не помогало и о месторасположении треда знали все, кому этого хотелось и кто был в силах. Однако эврикаскрипты перестали поддерживать двач и включить эту поддержку обратно с нахрапа не удалось и вообще оказалось делом довольно замысловатым.

После этого нагуглилась вакаба экстеншн, но не давала покоя возможность превью постов которая была на нольчане, да и, чего уж греха таить, идея написать собственный скрипт давно не давала покоя.

Итак, причин появления скрипта можно выделить две: недостаточный функцонал вакабы, при отсутствии нужных юзерскриптов и желание попогромировать.

\section*{История появления пеночки}

Пеночка впервые появилась в треде в /pr/ о том, что вот там-то и сям-то есть юзерскрипты, а на дваче нет. Появилась она там в виде заявления о том, что нет больше сил терпеть и скоро скрипты будут написаны и последующего скриншота самой первой альфа-версии, которая еще не использовала jQuery и в паблике была представлена одним лишь скриншотом.

Самая первая альфа-версия была достаточно трудна в разработке для тех, кто ее разрабатывал, так как опыта программирования на яваскрипте было очень недостаточно, а желания и умения луркать в гугле также не так много, поэтому была создана вторая версия, с активным использованием библиотеки jQuery которая разрабатывается и по сей день.

Первый тред о том со второй версией пеночки появился не в /s/, а в /b/ и был быстро смыт, захватив тем не менее, пару-тройку пользователей. А потом и появились треды в /s/. Человек, который разрабатывал скрип создал лишь первый тред, остальные создавались автоматически. 

К процессу разработки присоединился некий dvanon, который создал страничку на userscripts.org, по внутренним ощущениям несколько раз пиарил скрип (хотя это лишь внутренние предположения и верить им не стоит), и дал скрипту новый бренд govno 3, который используется и по сей день. Не смотря на частенько обрущивающуюся на дванона критику (в основном о том, что он стал причиной того, что эврикаскрипты не поддерживают двач, хотя очевидно, что виновника тут два) его роль в разработке пеночки более положительна, чем отрицательна: дело в том, что дванон использует firefox неиспользуемый разработчиками и поэтому версия дла этого браузера на userscripts.org как правило появляется отлаженной, в то время как от релизов в тредах /s/ можно ожидать всякого.

\section*{О том, кто истинный автор пеночки}

При всем желании некоторых доказать всем обратное, версия под брендом govno 3, в том числе и распространяемая на userscripts.org вторична: пеночка использует СКВ git, и имеет репозиторий в интернете на сайте github.com, который и является первоисточником распространения. А версии govno 3 получаются из пеночки путем некоторых переименований. Яваскрипт в них должен быть один и тот же, хотя на 100\% гарантировать соответствие нельзя, потому что на userscripts.org скрип заливается другим человеком, который гипотетически может что-то поменять. На практике таких прецедентов пока не происходило, но заботящимся о безопасности все равно следует проводить аудит исходного кода: никому доверять нельзя. 

А по сути и не важно, кто кого и кем считает, покуда это не влияет на процесс разработки, а пока это как раз и не влияет --- разработчик может разрабатывать скрип даже если его и не считают разработчиком. Посему нет нужды уделять излишнее внимание авторству скрипта, хотя желающие могут проситать лицензию, под которой выпускается скрипт: автор там четко прописан. Итак, об этом сказано.

\section*{jQuery и быстродействие}

Использование jQuery несомненно имеет некоторые недостатки, однако важнее оказалось следующее её приемущество --- она позволяет неискушенному яваскрипт-разработчику стать более искушенным в гораздо более которкие сроки, нежли если бы он использовал простой яваскрипт. Это и неудивительно, потому что jQuery создавалась с нуля, и ей не было необходимости поддерживать обратную совместимость со старыми api, чем страдают все хоть сколько-нибудь долгоживущие технологии.

Вопреки сложившемуся мнению jQuery влияет на быстродействие скрипта не столь драматическим образом. Основные причины низкого быстродействия скрипта --- неэффективные базовые алгоритмы, переписать которые сейчас довольно трудно, хотя попытки этого изредка и предпринимаются.

\section*{Состояние на сегодняшний день и перспективы развития}

На сегодняшний день пеночка страдает от нехватки времени. Хотя разработка более или менее тлеет: багрепорты докладываются, и иногда даже исправляются баги и добавляются новые возможности. Однако надо понимать, что пеночка как и любой другой проект делается по мере сил, и не планирует переходить на какую-либо из коммерческих моделей разработки: есть очень много более приятных и полезных работ, чем разработка юзерскрипта за деньги.

Что может радовать, так это то, что есть еще очень много вещей, которые хотелось бы реализовать, и ситуации когда <<все вроде бы отлажено и работает>> в ближайшем будущем не предвидится. Планов хватит ещё лет на пятьдесят, в течение которых если и не пользователи скрипта (хотя есть надежда, что и они), то хотя бы разработчик проведут часть своей жизни с пользой и веселием.

\section*{Благодарности\ldots}

\ldots выражаются всем, кто хоть каким-либо образом помогал, мешал процессу разработки и вообще всячески веселил, ну вы понели.
\end{document}