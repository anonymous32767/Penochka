\documentclass[12pt,a4paper]{article}
\usepackage[T2A]{fontenc}
\usepackage[utf8]{inputenc}
\usepackage[russian]{babel}
\title{Замечания к дизайну больших юзерскриптов}
\date{\today\\
\small Черновая версия}
\begin{document}
\maketitle

\begin{abstract}
В этом тексте я попытался поделиться опытом разработки пользовательского яваскрипта (user javascript, greasemonkey script) большой сложности, каковым является пеночка. Все написанное здесь следует воспринимать критически: пеночка не лучший образчик юзерскрипта, стало быть и в моих словах доля истины бывает не столь велика, сколь хотелосьбы. Я пишу это не пытаясь стать гурой и насаждать всюду свое мнение, но с надеждой, что этот текст будет прочитан, понят, и, возможно, критики укажут на мои заблуждения и ошибки. Неискушенным же разработкой больших проектов, тем, кто только берется или планирует браться за это дело, прочитать эту бумажку будет только полезно.

Некоторым возможно покажется, что слово <<сложный>> не совсем правильно квалифицирует пеночку и те примеры, что я привожу. Вполне вероятно, ведь я отталкивался от собственных представлений о сложности проекта, а приводить собственные представления в соотвествие с общепринятыми дело конечно полезное, но слишком жирное для одного маленького документика. 

Замечания ориентированы в большей степени на тех, кто желает писать свои юзерскрипты с нуля, потому что я поступил именно так. Вопросы адекватности этого шага оставим в стороне, но я ни чуть сожалею о содеяном и считаю, что изобретать велосипед в этом случае можно и нужно: все известные мне юзерскрипты имеют не бог весть какой дизайн, так что сделать сильно лучше, порвав зад при этом не слишком сильно, по-прежнему вполне реально.

Ребятам читавшим книжку С. Макконнелла <<Совершенный код>> этот текст принесет мало полезного.
\end{abstract}

\section{Лучшим юзерскриптом является его отсутствие}

Прежде чем браться за написание юзерскрипта надо понимать, что такие скрипты являются костылями по своему происхождению и природе. Если память не подводит, то изначально они создавались более для того, чтобы починить некоторые сайты, заточенние только под ИЕ, сделать их хоть как-то отображаемыми в альтернативных браузерах (файлик browser.js в пользовательской директории оперы до сих пор находится на своем законном месте и сваливать оттуда не собирается). 

Используя юзерскрипт вы почти всегда делаете простые по функционалу вещи сложными способами, потому что сложность яваскрипта зависит от того, с чем он реагирует и прежде всего от хтмл сайта, а он как правило не заточен под те проблемы, которые вы планируете решать. Поэтому лучший яваскрипт --- это всегда родной яваскрипт сайта. И поэтому же, лучшим первым шагом, который вы можете сделать, готовясь писать яваскрипт --- это связаться с администрацией сайта попробовать договориться с ней. 

Да обильный функционал нравится не всем. Верно и то, что некоторые функции плохо подойдут тем, кто сидит с разного рода мобильных устройств. Но всякий яваскрипт можно сделать отключаемым, и грузить, к примеру, только при наличии определенной куки. Этот шаг имеет ещё один крупный недостаток: ваши интересы должны довольно-таки сильно пересекаться с интересами администрации --- вы должны чувствовать друг в друге надежность, допускающую долговременное сотрудничество. Такое, разумеется, бывает не всегда. Наконец, фактически невозможным будет разместить яваскрипт на сервере, если вы планируете написать что-нибудь мультибордовое.

Но если местный зой вполне хорош, а вы не планируете ничего глобального --- попытайтесь договориться. Убивать за это вас никто не станет, а толку и выгоды от этого может получиться целый вагон. Административный резерв будет полезен.

\section{Существенные особенности большого юзерскрипта}

\subsection{Сложность}

Конечно же большой пользовательский яваскрипт, как и другой большой проект --- вещь прежде всего сложная и именно со сложностью вам придется бороться большее количество времени (как и в любом большом проекте). Суровая реальность такова, что компьютер может обработать сколько угодно данных, а человеческие мозги по-прежнему эффективно управляются с 5-9 объектами, а для больших объемов приходится использовать костыли.

Поглядите на код вордпресса, на код линукса, на исходные тексты наутилуса. Поглядите на продукты микрософт: шарепоинт, актив директори, аксапту, тот же офис. Все они содержат n-ое количество бюрократии: ненужных, казалось бы, вещей. Это последствия коммандной работы программистов и попытки уйти от сложности.

Нечто подобное будет и у вас, более того, ваш код будет гораздо хуже если вы не подумаете архитектуру с самого начала, не ограничите себя некиими правилами и не будете строго их соблюдать. Смекалка, луркоебство и прочие навыки, полезные и решающие при написании скриптиков на баше отходят на второй план, хотя и важны. А на первый план выходит дисциплинированность. Если хочется по-быстрому написать времянку, чтобы <<посмотреть как работает>>, то лучше этого не делать. Или найдите в себе силы написать сразу хорошо, или не пишите вовсе. Очевидная, вообщем-то вещь.

Хотя, с другой стороны, юзерскрипт не такая уж и сложная и замысловатая вещь и если вы видите что выходит какая-то слишком замысловатая хрень, то в 80\% случаев её можно переписать проще. Но все равно, готовьтесь к худщему. Мало ли куда вас заведет процесс разработки.

\subsection{Изменчивость окружающей среды}

То, на что закладывается ваш юзерскрипт, --- хтмл имиджборды или какого другого сайта имеют обыкновение меняться вас не спросясь. Имеют обыкнование также меняться и расположения разного рода файлов, которые нужны юзерскрипту, к примеру флеш-плеера. Поэтому и ваш скрипт должен быть легко изменяемым и конфигурируемым, хотя вообще он всегда должен быть таким: все твердое и негибкое крошится и гибнет в мире под луной.

\subsection{Особенность большого юзерскрипта, которой он должен обладать, но которая отсутствует во всех реализациях}

Это конечно модульность, то есть средства сокрытия информации. Именно это, стало одной из причин того, что разработка пеночки идет все медленнее. Именно это стало одной из причин того, что на сегодняшний день у нас нет юзерскрипта, в исходный код которого стабильно комитило бы более 1 человека.

Было бы очень приятно увидеть новый яваскрипт, где была бы сначала задокументирована архитектура, описан открытый интерфейс, и код соответствовал всему этому делу. К сожалению то ли это слишком сложно, то ли слишком затратно, то ли никому особенно не нужно.

Конкретных советов по модульности дать сложно: для него она в большинстве случаев не актуальна, поэтому никто над этим особо не задумывался. Для начала части, отдельные по функционалу можно заключить в конструкции вида:

\begin{verbatim}
;(function () {
    do smth private here...
})();
\end{verbatim}

и продумать механизм перекидывания сообщениями между этими импровизированными атомами. Встроенные сообщения ({\tt addEventListener}) лучше не использовать --- будут тормоза и они не для этого нужны.

\subsection{Советы начинающему разработчику}

\begin{enumerate}
\item Если вы беретесь за дело плохо зная яваскрипт, то лучше а) сначала подучить яваскрипт и б) писать на основе какого-либо фреймворка. Так вы быстрее войдете в тему и потратите меньше сил. Конечно скрипт, возможно, будет и не так хорош, как написаный на голом js и страшно оптимизированный.

\item Если вы не слишком аккуратны, трижды продумайте сначала архитектуру, прежде чем начать писать. Научитесь пользоваться СКВ, кроме того это наверное обязательно при коммандной разработке. 

\item Заведите багтреккер. Очень хорошо, если это будет багтреккер, куда можно отправлять баги анонимно: замечания, оставляемые пользователями в тредах проебываются с завидной периодичностью, что не может не отражаться на дружественной атмосфере в тредах.

\item Старайтесь работать над скриптом равномерно, а не <<запоями>>: так меньше времени будет уходить на то, чтобы вспомнить все это дело.

\item Не оставляйте работу над скриптом с недоделанным кодом: иначе потом вас ожидает много радости и веселья в поисках того, где скрипт работает не так.
\end{enumerate}

\section{Предупреждение разработчикам или социальный аспекты ведения больших проектов на АИБ}

Человек, долгое время занимающийся чем-то одним на АИБ теряет анонимность. На него обращается повышенное внимание участников дискуссии, он становится узнаваем с той или иной долей вероятности по тому тексту, что пишет: по характерным оборотам, фразеологизмам. По картинкам, которые постит.

Прежде чем заняться чем-то подобным, подумайте за тем ли вы пришли на АИБ и готовы ли вы ради некоторых приемуществ терять свою анонимность, то зачем мы здесь все находимся.
\end{document}
